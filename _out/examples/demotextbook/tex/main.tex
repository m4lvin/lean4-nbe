
\documentclass{memoir}

\usepackage{sourcecodepro}
\usepackage{sourcesanspro}
\usepackage{sourceserifpro}

\usepackage{fancyvrb}
\usepackage{fvextra}

\makechapterstyle{lean}{%
\renewcommand*{\chaptitlefont}{\sffamily\HUGE}
\renewcommand*{\chapnumfont}{\chaptitlefont}
% allow for 99 chapters!
\settowidth{\chapindent}{\chapnumfont 999}
\renewcommand*{\printchaptername}{}
\renewcommand*{\chapternamenum}{}
\renewcommand*{\chapnumfont}{\chaptitlefont}
\renewcommand*{\printchapternum}{%
\noindent\llap{\makebox[\chapindent][l]{%
\chapnumfont \thechapter}}}
\renewcommand*{\afterchapternum}{}
}

\chapterstyle{lean}

\setsecheadstyle{\sffamily\bfseries\Large}
\setsubsecheadstyle{\sffamily\bfseries\large}
\setsubsubsecheadstyle{\sffamily\bfseries}

\renewcommand{\cftchapterfont}{\normalfont\sffamily}
\renewcommand{\cftsectionfont}{\normalfont\sffamily}
\renewcommand{\cftchapterpagefont}{\normalfont\sffamily}
\renewcommand{\cftsectionpagefont}{\normalfont\sffamily}

\title{\sffamily Normalization by Evaluation in Lean4}
\author{\sffamily Jeremy Sorkin}
\date{\sffamily }

\begin{document}

\frontmatter

\begin{titlingpage}
\maketitle
\end{titlingpage}

\tableofcontents

\mainmatter

\chapter{Introduction}

For this project, I chose to formalize Peter Dybjer's \hyperlink{"https://www.cs.le.ac.uk/events/mgs2009/courses/courses.html#anchorPeter"}{Normalization-by-Evaluation} slides into Lean4.
My motivation for doing this was for 2 primary reasons:

\begin{itemize}
\item It's an interesting normalization-strategy for rewriting systems that leverages on the interplay between the "Object" and "Meta"-levels in a nontrivial manner.\item It also provides a nice oppurtunity for me to further sharpen and apply my Lean4 skills
\end{itemize}


The goal of this document is to walk the reader through both Dybjer's slides and their Lean4 formalization in a step-by-step manner.
# Motivation

Normalization by Evaluation is a technique whereby we can normalize terms of a given rewrite-system by "evaluating" them into the Meta-level (where normalization occurs)
followed by "reifying" our normalized term back into the Object-level. In this way, we will be translating our "reduction-based" rewriting-relation into a "reduction-free" equality check
that will let us show, among other results, decidability of rewriting.
As a normalization technique, has a history of application to different type systems:

\begin{itemize}
\item \hyperlink{"https://www.mathematik.uni-muenchen.de/~schwicht/papers/lics91/paper.pdf"}{Berger and Schwichtenberg (1991)} utilize Nbe to give a normalization-procedure for Simple Type Theory with βη-rewriting.\item \hyperlink{"https://www.mathematik.uni-muenchen.de/~logik/minlog/"}{MINLOG proof-assistant} utilizes Nbe as a normalization procedure for minimal first order logic.\item \hyperlink{"https://www.cse.chalmers.se/~peterd/papers/GlueingTypes93.pdf"}{Coquand and Dybjer 1993} utilize Nbe to give a decision algorithm for Combinatory logic as well as implement its
    formal correctness-proof in the ALF proof-assistant.\item \hyperlink{"https://tidsskrift.dk/brics/article/view/21870"}{Filinski and Rohde 2005} extended NbE to the Untyped Lambda-Calculus using an infinitary-variant of normal-forms, Bohm Trees.\item \hyperlink{"https://www.cse.chalmers.se/~peterd/papers/NbeMLTTEqualityJudgements.pdf"}{Abel, Dybjer, Coquand 2007} have extended the technique to Martin-Loff Type Theory.
\end{itemize}


For our purposes, we will focus on the simpler-examples of rewriting in Monoids and a Combinatory-version of Godel's System T.




\chapter{Monoid-Rewriting}



\section{Monoid Expressions and rewriting}

Let’s start with the simple case where we are rewriting in a Monoid.
Our Monoid-Expressions are:

\begin{verbatim}
inductive Exp (α : Type)
| app : Exp α → Exp α → Exp α
| id  : Exp α
| elem : α → Exp α
infix : 100 " ⬝ " => Exp.app

\end{verbatim}



That is, an expression is either:

\begin{itemize}
\item Applying two expressions together: \Verb|e₁ ⬝ e₂|\item The identity element \Verb|id|\item An element of \Verb|α|
\end{itemize}


We know that a Monoid has an identity element and is associative, so we naturally get
the following rewrite-relation \Verb|~|:

\begin{verbatim}
inductive convr : (Exp α) → (Exp α) → Prop
| assoc         : convr ((e ⬝ e') ⬝ e'') (e ⬝ (e' ⬝ e''))
| id_left       : convr ((Exp.id) ⬝ e) (e)
| id_right      : convr (e ⬝ Exp.id) (e)
| refl          : convr (e) (e)
| sym           : convr (e) (e') → convr (e') (e)
| trans         : convr (e) (e') → convr (e') (e'') → convr (e) (e'')
| app           : convr (a) (b) → convr (c) (d) → convr (a ⬝ c) (b ⬝ d)
infix : 100 " ~ " => convr

\end{verbatim}



The first 3 constructors are the normal Monoid-laws, the next 3 ensure \Verb|~| is an equivalence-relation, and the final ensures it is congruent wrt \Verb|app|.
This gives us the normal Monoid-behavior we expect as the following examples show:

\begin{verbatim}
def zero := (Exp.id : Exp Nat)
def one  := Exp.elem 1
def two  := Exp.elem 2
def three := Exp.elem 3

-- (1 ⬝ 2) ⬝ ((0 ⬝ 0) ⬝ 3) ~ 1 ⬝ (2 ⬝ (3 ⬝ 0))
theorem example1
        : ((one ⬝  two) ⬝ ((zero ⬝  zero) ⬝  three))
        ~ (one ⬝  (two ⬝  (three ⬝  zero))) :=
  by
  -- Hint:
  -- (1 ⬝ 2) ⬝ ((0 ⬝ 0) ⬝ 3) ~ (1 ⬝ 2) ⬝ (0 ⬝ 3)
  -- (1 ⬝ 2) ⬝ (0 ⬝ 3) ~ (1 ⬝ 2) ⬝ 3
  -- (1 ⬝ 2) ⬝ 3 ~ (1 ⬝ 2) ⬝ (3 ⬝ 0)
  -- (1 ⬝ 2) ⬝ (3 ⬝ 0) ~ 1 ⬝ (2 ⬝ (3 ⬝ 0))
  sorry

\end{verbatim}



\hyperlink{"https://live.lean-lang.org/#codez=JYWwDg9gTgLgBAWQIYwBYBtgCMB0AVJAYxmEICgzgA7AEwFdjgA3AUzgFEAPMOACiXRhUSOAC44eAJ5gWASjIAfOEjA9xXHgKEjASYQduywcLh6Nh7YrjAacMfs1GklluhYg7W46YOenlWgwkrHCEEFRMUHa8Zr6yJnwxjnF6AApQEGCWSADO2RCEcADebCwA5HBl5eo+jgC+dqHhkby8LDgqPGWy7aoVpaVxrT08raXDfQPyStYA+q4AZvDFttUO2vW2KyFhEXzR3DjW3R0Vg3KWs1DAAOaoS2yr5sIbm+KNu0MnGoc0Z1NwUBY83Qm1sy0evheW3ezTkfHOSmykncoKKJSq9ieSA2bx2sLOA3iMPhhNa/xgUCQVGyaL6EzsiXWDTx8IJyW2TRJBMJemJZJJAyyvUKIiwITgNghdVBuM5/EGWHZfMIg1+RJZ/HGKr4uBOv3882AnDsAEYAAxmuAAIjgAD9rXAALwAPg5EQoZBoQLgAC8WOkxI6EgdrAyDIBUQnkXvmcDCbED9hwLjccBNnu9MAA7hAVkHvsn3AAmdMxtCAh55g4FuAAZgoAFp63wTXBALjUcELcXbLUt7bNXdrcXtLe7hbbfBr4/78lcIBAIhYnCQ4FcLbIqNefDj4yzEGOvT2fvS4yPe53qHL8g3tntvG3J14u/Gj4vLDaJ1Psi/gfXcCwkl/Rs4AACWoGBREApteBHDsBx7KcBxrIdm3HTtx14XtB0glD2zQ7tMKQu0cNg8c61sIDoNQxCiMo3C4MnPsrzgCiYLwicEKImDeDHbsGLgadfzyKAoEkIA"}{Try it!}
\hyperlink{"https://live.lean-lang.org/#codez=JYWwDg9gTgLgBAWQIYwBYBtgCMB0AVJAYxmEICgzgA7AEwFdjgA3AUzgFEAPMOACiXRhUSOAC44eAJ5gWASjIAfOEjA9xXHgKEjASYQduywcLh6Nh7YrjAacMfs1GklluhYg7W46YOenlWgwkrHCEEFRMUHa8Zr6yJnwxjnF6AApQEGCWSADO2RCEcADebCwA5HBl5eo+jgC+dqHhkby8LDgqPGWy7aoVpaVxrT08raXDfQPyStYA+q4AZvDFttUO2vW2KyFhEXzR3DjW3R0Vg3KWs1DAAOaoS2yr5sIbm+KNu0MnGoc0Z1NwUBY83Qm1sy0evheW3ezTkfHOSmykncoKKJSq9ieSA2bx2sLOA3iMPhhNa/xgUCQVGyaL6EzsiXWDTx8IJyW2TRJBMJemJZJJAyyvUKIiwITgNghdVBuM5/EGWHZfMIg1+RJZ/HGKr4uBOv3882AnDsAEYAAxmuAAIjgAD9rXAALwAPg5EQoZBoQLgAC8WOkxI6EgdrAyDIBUQnkXvmcDCbED9hwLjccBNnu9MAA7hAVkHvsn3AAmdMxtCAh55g4FuAAZgoAFp63wTXBALjUcELcXbLUt7bNXdrcXtLe7hbbfBr4/78lcIBAIhYnCQ4FcLbIqNefDj4yzEGOvT2fvS4yPe53qHL8g3tntvG3J14u/Gj4vLDaJ1Psi/gfXcCwkl/YRglQS1xF4U8T39Pc7V9KCExhH45iBGBAKQYCWzA8CoMg9J9x4Ms3yHPgIJOAi4VEIMEJOEC3SgHBAWBVDgLHTD716Xc8MPbCPygziyK/X9rxglo2Pw7NOKw49SNfOQ4go2itTxeigRBVA01sIC2FQSdWKod92PE58SPYmSBKEm89lEuAOPPS94KUk4EIYkFeAQ2YFhgK84E07yABYojvPSd0M6S7ME69b0C/SxLPB8yJwvc5MohzeiclS+DcmgZiuW4YBwJEQDiJitIAVgCqybLimSErM8zbysx9s2feKeNw795Ko3J8l/QhXCQSIaLU39f0bZtx07cceynAcayI3gRw7AdeF7Qdfw6dBJAUikqRpVAxxGpt5vGpaVtm4SFom9s61sdbNoQ7bqW8664FGo720u2tzuOybJz7Lzbq2ylHtQXyDrG96lt+uB+xghbeDHbsoenNbVA2wGdu8krf0XIh4DS4EgA"}{Solution}

\begin{verbatim}
-- (0 ⬝ (1 ⬝ 0)) ⬝ ((0 ⬝  2) ⬝ (3 ⬝ 0)) ~ 1 ⬝ (2 ⬝ (3 ⬝ 0))
theorem example2
  : ((zero ⬝ (one ⬝ zero))  ⬝  ((zero ⬝ two) ⬝ (three ⬝ zero)))
  ~ (one ⬝ (two ⬝ (three ⬝ zero))) :=
  by
  -- Hint:
  -- (0 ⬝ (1 ⬝ 0)) ⬝ ((0 ⬝  2) ⬝ (3 ⬝ 0)) ~ (1 ⬝ 0) ⬝ ((0 ⬝  2) ⬝ (3 ⬝ 0))
  -- (1 ⬝ 0) ⬝ ((0 ⬝  2) ⬝ (3 ⬝ 0)) ~ 1 ⬝ ((0 ⬝  2) ⬝ (3 ⬝ 0))
  -- 1 ⬝ ((0 ⬝  2) ⬝ (3 ⬝ 0)) ~ 1 ⬝ (2 ⬝ (3 ⬝ 0))
  sorry

\end{verbatim}



\hyperlink{"https://live.lean-lang.org/#codez=JYWwDg9gTgLgBAWQIYwBYBtgCMB0AVJAYxmEICgzgA7AEwFdjgA3AUzgFEAPMOACiXRhUSOAC44eAJ5gWASjIAfOEjA9xXHgKEjASYQduywcLh6Nh7YrjAacMfs1GklluhYg7W46YOenlWgwkrHCEEFRMUHa8Zr6yJnwxjnF6AApQEGCWSADO2RCEcADebCwA5HBl5eo+jgC+dqHhkby8LDgqPGWy7aoVpaVxrT08raXDfQPyStYA+q4AZvDFttUO2vW2KyFhEXzR3DjW3R0Vg3KWs1DAAOaoS2yr5sIbm+KNu0MnGoc0Z1NwUBY83Qm1sy0evheW3ezTkfHOSmykncoKKJSq9ieSA2bx2sLOA3iMPhhNa/xgUCQVGyaL6EzsiXWDTx8IJyW2TRJBMJemJZJJAyyvUKIiwITgNghdVBuM5/EGWHZfMIg1+RJZ/HGKr4uBOv3882AnDsAEYAAxmuAAIjgAD9rXAALwAPg5EQoZBoQLgAC8WOkxI6EgdrAyDIBUQnkXvmcDCbED9hwLjccBNnu9MAA7hAVkHvsn3AAmdMxtCAh55g4FuAAZgoAFp63wTXBALjUcELcXbLUt7bNXdrcXtLe7hbbfBr4/78lcIBAIhYnCQ4FcLbIqNefDj4yzEGOvT2fvS4yPe53qHL8g3tntvG3J14u/Gj4vLDaJ1Psi/gfXcCwkl/YRglQS1xF4U8T39Pc7V9KCExhH45iBGBAKQYCWzA8CoMg9J9x4Ms3yHPgIJOAi4VEIMEJOEC3SgHBAWBMhG2bcdO3HHspwHGsiN4EcOwHXhe0HVDgLHTD716Xc8MPbCPyg6SyK/X9rxgloJPw7NpKw49SNfOQ4go2itTxeigRBVA02Y3jWIEoTuNUvi2PbOtbCAthUEncSqHfSTNOfEjJL0pSVJvPZ1LgKTz0veCTJOBCGJBXgENmBYYHkKzHK4hybPYyc+yvOA3MKgAWKI728nc/N06LlOvW9yp8jSzwfMicL3AzKNi3p4rMvhkpoGYrluGAcCREA4iYptrPbJyJ04mC+N4MduzyuBpxE9yAFYyvCyKWr0trgpC29wsfbNn1auTcO/QyqNyfJf0IVwkEiGiLN/Dp0EkIyKSpGlUDHD7VC+n7KWpQqXOUYHvoQ37wdQYqgbAEHYbB/7Nt/RciHgHrGMmvghOmtbv27QTx343LOKIxblrm/KZzcecKiXFcWGLLZtIgZ9ws/LTOcq5rehfctDoKk6Kpa879pFq72o639/1/ZiAAlqBgUQlamwm+OndiyfbCmVqp7L8r1oTDbp4mCoy+bSfN2beFW3XhzN8mHado7mMW/XbHd42XdHSn6d/PIoCgSQgA"}{Try it!}
\hyperlink{"https://live.lean-lang.org/#codez=JYWwDg9gTgLgBAWQIYwBYBtgCMB0AVJAYxmEICgzgA7AEwFdjgA3AUzgFEAPMOACiXRhUSOAC44eAJ5gWASjIAfOEjA9xXHgKEjASYQduywcLh6Nh7YrjAacMfs1GklluhYg7W46YOenlWgwkrHCEEFRMUHa8Zr6yJnwxjnF6AApQEGCWSADO2RCEcADebCwA5HBl5eo+jgC+dqHhkby8LDgqPGWy7aoVpaVxrT08raXDfQPyStYA+q4AZvDFttUO2vW2KyFhEXzR3DjW3R0Vg3KWs1DAAOaoS2yr5sIbm+KNu0MnGoc0Z1NwUBY83Qm1sy0evheW3ezTkfHOSmykncoKKJSq9ieSA2bx2sLOA3iMPhhNa/xgUCQVGyaL6EzsiXWDTx8IJyW2TRJBMJemJZJJAyyvUKIiwITgNghdVBuM5/EGWHZfMIg1+RJZ/HGKr4uBOv3882AnDsAEYAAxmuAAIjgAD9rXAALwAPg5EQoZBoQLgAC8WOkxI6EgdrAyDIBUQnkXvmcDCbED9hwLjccBNnu9MAA7hAVkHvsn3AAmdMxtCAh55g4FuAAZgoAFp63wTXBALjUcELcXbLUt7bNXdrcXtLe7hbbfBr4/78lcIBAIhYnCQ4FcLbIqNefDj4yzEGOvT2fvS4yPe53qHL8g3tntvG3J14u/Gj4vLDaJ1Psi/gfXcCwkl/YRglQS1xF4U8T39Pc7V9KCExhH45iBGBAKQYCWzA8CoMg9J9x4Ms3yHPgIJOAi4VEIMEJOEC3SgHBAWBMhG2bcdO3HHspwHGsiN4EcOwHXhe0HVDgLHTD716Xc8MPbCPyg6SyK/X9rxgloJPw7NpKw49SNfOQ4go2itTxeigRBVA02Y3jWIEoTuNUvi2PbOtbCAthUEncSqHfSTNOfEjJL0pSVJvPZ1LgKTz0veCTJOBCGJBXgENmBYYHkKzHK4hybPYyc+yvOA3MKgAWKI728nc/N06LlOvW9yp8jSzwfMicL3AzKNi3p4rMvhkpoGYrluGAcCREA4iYptrPbJyJ04mC+N4MduzyuBpxE9yAFYyvCyKWr0trgpC29wsfbNn1auTcO/QyqNyfJf0IVwkEiGiLN/Dp0EkIyKSpGlUDHD7VC+n7KWpQqXOUYHvoQ37wdQYqgbAEHYbB/7Nt/RciHgHrGMmvghOmtbv27QTx343LOKIxblrm/KZzcecKiXFcWGLLZtIgZ9ws/LTOcq5rehfctDoKk6Kpa879pFq72o639/w2wrQOI2ShZ5+S+YCpqtMu3pedqjdxca2DcOffnSKqoW9Z4XmYqaZ9+qQxY4lx9AlYsqINbNh8Ld8wWRht032sN1F7VOv2dYug7ZZuzqHYfJ2hruV2TISj2xNjCWhcjiKrcDmP9c10PQXD7PA6l63C9tzX7YiYyHYSvqup4J3UqMhKCqswm+OndiyfbCmVqp7L8v7oSh7p4mCs+mGTLh/6zV/DL5tJifZt4Va++HcfyY3rejtn0G/sKtNbGYxaB9sfeR530dKfppGUfntHCvZ5nsY7sygA"}{Solution}

\begin{verbatim}
-- (1 ⬝ 2) ⬝ ((0 ⬝ 0) ⬝ 3) ~ (0 ⬝ (1 ⬝ 0)) ⬝ ((0 ⬝  2) ⬝ (3 ⬝ 0))
theorem example3
  : ((one ⬝ two) ⬝ ((zero ⬝ zero) ⬝ three))
  ~ ((zero ⬝ (one ⬝ zero)) ⬝ ((zero ⬝ two) ⬝ (three ⬝ zero))) :=
  by
  -- Hint: Use examples 1 and 2!
  sorry

\end{verbatim}



\hyperlink{"https://live.lean-lang.org/#codez=JYWwDg9gTgLgBAWQIYwBYBtgCMB0AVJAYxmEICgzgA7AEwFdjgA3AUzgFEAPMOACiXRhUSOAC44eAJ5gWASjIAfOEjA9xXHgKEjASYQduywcLh6Nh7YrjAacMfs1GklluhYg7W46YOenlWgwkrHCEEFRMUHa8Zr6yJnwxjnF6AApQEGCWSADO2RCEcADebCwA5HBl5eo+jgC+dqHhkby8LDgqPGWy7aoVpaVxrT08raXDfQPyStYA+q4AZvDFttUO2vW2KyFhEXzR3DjW3R0Vg3KWs1DAAOaoS2yr5sIbm+KNu0MnGoc0Z1NwUBY83Qm1sy0evheW3ezTkfHOSmykncoKKJSq9ieSA2bx2sLOA3iMPhhNa/xgUCQVGyaL6EzsiXWDTx8IJyW2TRJBMJemJZJJAyyvUKIiwITgNghdVBuM5/EGWHZfMIg1+RJZ/HGKr4uBOv3882AnDsAEYAAxmuAAIjgAD9rXAALwAPg5EQoZBoQLgAC8WOkxI6EgdrAyDIBUQnkXvmcDCbED9hwLjccBNnu9MAA7hAVkHvsn3AAmdMxtCAh55g4FuAAZgoAFp63wTXBALjUcELcXbLUt7bNXdrcXtLe7hbbfBr4/78lcIBAIhYnCQ4FcLbIqNefDj4yzEGOvT2fvS4yPe53qHL8g3tntvG3J14u/Gj4vLDaJ1Psi/gfXcCwkl/YRglQS1xF4U8T39Pc7V9KCExhH45iBGBAKQYCWzA8CoMg9J9x4Ms3yHPgIJOAi4VEIMEJOEC3SgHBAWBMhG2bcdO3HHspwHGsiN4EcOwHXhe0HVDgLHTD716Xc8MPbCPyg6SyK/X9rxgloJPw7NpKw49SNfOQ4go2itTxeigRBVA02Y3jWIEoTuNUvi2PbOtbCAthUEncSqHfSTNOfEjJL0pSVJvPZ1LgKTz0veCTJOBCGJBXgENmBYYHkKzHK4hybPYyc+yvOA3MKgAWKI728nc/N06LlOvW9yp8jSzwfMicL3AzKNi3p4rMvhkpoGYrluGAcCREA4iYptrPbJyJ04mC+N4MduzyuBpxE9yAFYyvCyKWr0trgpC29wsfbNn1auTcO/QyqNyfJf0IVwkEiGiLN/Dp0EkIyKSpGlUDHD7VC+n7KWpQqXOUYHvoQ37wdQYqgbAEHYbB/7Nt/RciHgHrGMmvghOmtbv27QTx343LOKIxblrm/KZzcecKiXFcWGLLZtIgZ9ws/LTOcq5rehfctDoKk6Kpa879pFq72o639/w2wrQOI2ShZ5+S+YCpqtMu3pedqjdxca2DcOffnSKqoW9Z4XmYqaZ9+qQxY4lx9AlYsqINbNh8Ld8wWRht032sN1F7VOv2dYug7ZZuzqHYfJ2hruV2TISj2xNjCWhcjiKrcDmP9c10PQXD7PA6l63C9tzX7YiYyHYSvqup4J3UqMhKCqswm+OndiyfbCmVqp7L8v7oSh7p4mCs+mGTLh/6zV/DL5tJifZt4Va++HcfyY3rejtn0G/sKtNbGYxaB9sfeR530dKfppGUfntHCvZ5nsY7syGymzLd7H5yPEe4jzXnvASB8GZzgXCzZGLBIZgV2vnWwLRtbBwUkFX8t5UENVFubVBe0q4yyLtdH8thFbnybAACWoDAcQABVbIJQYGuBpC2KkNhCwAEJfx5CgFASQQA"}{Try it!}
\hyperlink{"https://live.lean-lang.org/#codez=JYWwDg9gTgLgBAWQIYwBYBtgCMB0AVJAYxmEICgzgA7AEwFdjgA3AUzgFEAPMOACiXRhUSOAC44eAJ5gWASjIAfOEjA9xXHgKEjASYQduywcLh6Nh7YrjAacMfs1GklluhYg7W46YOenlWgwkrHCEEFRMUHa8Zr6yJnwxjnF6AApQEGCWSADO2RCEcADebCwA5HBl5eo+jgC+dqHhkby8LDgqPGWy7aoVpaVxrT08raXDfQPyStYA+q4AZvDFttUO2vW2KyFhEXzR3DjW3R0Vg3KWs1DAAOaoS2yr5sIbm+KNu0MnGoc0Z1NwUBY83Qm1sy0evheW3ezTkfHOSmykncoKKJSq9ieSA2bx2sLOA3iMPhhNa/xgUCQVGyaL6EzsiXWDTx8IJyW2TRJBMJemJZJJAyyvUKIiwITgNghdVBuM5/EGWHZfMIg1+RJZ/HGKr4uBOv3882AnDsAEYAAxmuAAIjgAD9rXAALwAPg5EQoZBoQLgAC8WOkxI6EgdrAyDIBUQnkXvmcDCbED9hwLjccBNnu9MAA7hAVkHvsn3AAmdMxtCAh55g4FuAAZgoAFp63wTXBALjUcELcXbLUt7bNXdrcXtLe7hbbfBr4/78lcIBAIhYnCQ4FcLbIqNefDj4yzEGOvT2fvS4yPe53qHL8g3tntvG3J14u/Gj4vLDaJ1Psi/gfXcCwkl/YRglQS1xF4U8T39Pc7V9KCExhH45iBGBAKQYCWzA8CoMg9J9x4Ms3yHPgIJOAi4VEIMEJOEC3SgHBAWBMhG2bcdO3HHspwHGsiN4EcOwHXhe0HVDgLHTD716Xc8MPbCPyg6SyK/X9rxgloJPw7NpKw49SNfOQ4go2itTxeigRBVA02Y3jWIEoTuNUvi2PbOtbCAthUEncSqHfSTNOfEjJL0pSVJvPZ1LgKTz0veCTJOBCGJBXgENmBYYHkKzHK4hybPYyc+yvOA3MKgAWKI728nc/N06LlOvW9yp8jSzwfMicL3AzKNi3p4rMvhkpoGYrluGAcCREA4iYptrPbJyJ04mC+N4MduzyuBpxE9yAFYyvCyKWr0trgpC29wsfbNn1auTcO/QyqNyfJf0IVwkEiGiLN/Dp0EkIyKSpGlUDHD7VC+n7KWpQqXOUYHvoQ37wdQYqgbAEHYbB/7Nt/RciHgHrGMmvghOmtbv27QTx343LOKIxblrm/KZzcecKiXFcWGLLZtIgZ9ws/LTOcq5rehfctDoKk6Kpa879pFq72o639/w2wrQOI2ShZ5+S+YCpqtMu3pedqjdxca2DcOffnSKqoW9Z4XmYqaZ9+qQxY4lx9AlYsqINbNh8Ld8wWRht032sN1F7VOv2dYug7ZZuzqHYfJ2hruV2TISj2xNjCWhcjiKrcDmP9c10PQXD7PA6l63C9tzX7YiYyHYSvqup4J3UqMhKCqswm+OndiyfbCmVqp7L8v7oSh7p4mCs+mGTLh/6zV/DL5tJifZt4Va++HcfyY3rejtn0G/sKtNbGYxaB9sfeR530dKfppGUfntHCvZ5nsY7syGymzLd7H5yPEe4jzXnvASB8GZzgXCzZGLBIZgV2vnWwLRtbBwUkFX8t5UENVFubVBe0q4yyLtdH8thFbn1/jlUBADBwLXYrTTeI8n5zwdgvD+rMz5wAvvQh+09VLAPpv/a+4CqbMOPuDLGrNCyjWRJjJcxAv7AiAA"}{Solution}



\section{Normalization of Monoid-Expressions}

From the examples above, we can see that showing \Verb|a ~ b| step-by-step can be rather tedious.
When checking this in practice, we simply preform all these steps simultaneously by:

"removing all the \Verb|id|'s, shuffling parentheses to the right, then checking for equality"

Can we implement this normalization strategy by interpreting our Monoid-Expressions in a clever way?



\section{Evaluation}

This is exactly what evaluation does, it interprets our expressions such that normalization happens automatically at the Meta-level.
We will do this by interpreting Monoid-Expressions as functions, the "intended"-meaning:

\begin{itemize}
\item \Verb|app| will be function-composition\item \Verb|id| will be the Identity-function\item \Verb|x| will be \Verb|λ e ↦ x ⬝ e|: Applying \Verb|x| to the left
\end{itemize}


This gives us our evaluation function:

\begin{verbatim}
def eval : (Exp α) → (Exp α → Exp α)
  --               (eval a) ∘ (eval b)
  | Exp.app a b => (λ e => eval a (eval b e))
  --               Identity function
  | Exp.id      => id
  --               λ e ↦ x ⬝ e`
  | Exp.elem x  => (λ e => (Exp.elem x) ⬝ e)

\end{verbatim}



Now, by interpreting Monoid-expressions as function-compositions at the Meta-level,
Lean will automatically normalize these compositions as the following shows:

\begin{verbatim}
-- fun e => Exp.elem 1 ⬝ (Exp.elem 2 ⬝ (Exp.elem 3 ⬝ e))
#reduce eval $ (one ⬝ two) ⬝ ((zero ⬝ zero) ⬝ three)

-- fun e => Exp.elem 1 ⬝ (Exp.elem 2 ⬝ (Exp.elem 3 ⬝ e))
#reduce eval $ (zero ⬝ (one ⬝ zero)) ⬝ ((zero ⬝ two) ⬝ (three ⬝ zero))

-- fun e => Exp.elem 1 ⬝ (Exp.elem 2 ⬝ (Exp.elem 3 ⬝ e))
#reduce eval $ (one ⬝ (two ⬝ (three ⬝ zero)))

\end{verbatim}



From the above examples, we see that the evaluations of

\Verb|(1 ⬝ 2) ⬝ ((0 ⬝ 0) ⬝ 3) ~ (0 ⬝ (1 ~ 0)) ⬝ ((0 ⬝ 2) ⬝ (3 ⬝ 0)) ~ 1 ⬝ (2 ⬝ (3 ⬝ 0))|

are all equal to: \Verb|λ e ↦ 1 ⬝ (2 ⬝ (3 ⬝ e))|.

That is, rewritable-terms give us equal evaluations:

\begin{verbatim}
-- a ~ b  → eval a = eval b
theorem convr_eval_eq {a b : Exp α} (h : a ~ b) : ∀ c, (eval a) c  = (eval b) c :=
    by
    -- Hint: Induction on h
    sorry

\end{verbatim}



\hyperlink{"https://live.lean-lang.org/#codez=JYWwDg9gTgLgBAWQIYwBYBtgCMB0AVJAYxmEICgyBaSuAMSghAC45UYYwBnJgeh8M45gEQjgCmYnmIDunATwBMABiUBOHgBMAnlgBWYqDxBJgAOwAKSdCE5jKUMQDdgtjTjAaAZhTMaArsTAjmJwAKIAHmBwABSAjcBwLHhaYGIAlGQAPnBIYFEsEVHxgEmEYZFwxaWFmXDAGnAJleXVYuhiIA0VBU1mnsDhDQCMKnAARHCAuNSjcAC8AHyVODlgFFQ0ACK1pgDk8BpihOhIDnBoIQDKYjAQtTWmnDBIpoQhntAnqC5wWpdkvgEkwTghAgpkcUAa0S6sVScBKkLK0NhcHMDGWWSQnE4IjgAG8QmItnACYT8giAL4NYGg8HRaIhSYEmEMrZbGF0iYxAkc4mpdJZWoAfVanngePqpMKFPq4qBILBMXhYCEGiZRLZaWqgqgwAA5mxcSEJeUpdKWFT5ezJgVleq+XAHJ50NL6mKjbETTLzTS0py7ZwtO1nQaiSTGu7PXLverWUivZyY3S7TAoI9OMGucSGlCKWbI774zCSnG6QmebG84n46zqktcUg4FggXA6m6PbnqTEkGysIXZR3ooQ2Sry/365NBzFG5MVb9TL1+iwhkopgA/KZzPtglZ7TxExxWCFQ3uK8pIo9kepZa21+uNjfRQDdwESZvMnAf63T907G2l0peFjcQYbrUF5wFekTiK07T9C+MRPiE97Wi0bRwOEqoahQ1DZHAa6Nkib5OvW0x7geWBkFBxiblAAoETRACOdYNlm5IxKgDT1rhMIsIAAERAgANJyX7ZDChD1MRn6kSJCTTKBWBaKBfyBCCrCgVY6ACjqEBWGmBxiEc2FkaBhAYiEGJYqJt5NnMoH1GYyYQM2NlwFheDas4VhOViUBQPJRkmTUGhCmIIrYdZzp2QwDZOS5bnAB5zpeT5FD1MZtgBQK2p6vARGzE5EUOWRzoxUEcXoJ50BJX5aUOoRL55aY9lRUVNCuSV8XSolvkpf5/rtJZSAZQKjbAKgYXSsV7llQlFVdUC/nJqmBlNgN1GNlgg2iSNAw1KgChjfUE2leV3mzalpm5Etol1AKcA3Ug60jUCgWPftQKtPpABUznZA9bGEM9qD1Y1YhOX4c4QOgdQEdFNAABJ2SwADCFX7DA6BaLc9yPCQKCmb92SmHU/0Co9x1JUAA"}{Try it!}
\hyperlink{"https://live.lean-lang.org/#codez=JYWwDg9gTgLgBAWQIYwBYBtgCMB0AVJAYxmEICgyBaSuAMSghAC45UYYwBnJgeh8M45gEQjgCmYnmIDunATwBMABiUBOHgBMAnlgBWYqDxBJgAOwAKSdCE5jKUMQDdgtjTjAaAZhTMaArsTAjmJwAKIAHmBwABSAjcBwLHhaYGIAlGQAPnBIYFEsEVHxgEmEYZFwxaWFmXDAGnAJleXVYuhiIA0VBU1mnsDhDQCMKnAARHCAuNSjcAC8AHyVODlgFFQ0ACK1pgDk8BpihOhIDnBoIQDKYjAQtTWmnDBIpoQhntAnqC5wWpdkvgEkwTghAgpkcUAa0S6sVScBKkLK0NhcHMDGWWSQnE4IjgAG8QmItnACYT8giAL4NYGg8HRaIhSYEmEMrZbGF0iYxAkc4mpdJZWoAfVanngePqpMKFPq4qBILBMXhYCEGiZRLZaWqgqgwAA5mxcSEJeUpdKWFT5ezJgVleq+XAHJ50NL6mKjbETTLzTS0py7ZwtO1nQaiSTGu7PXLverWUivZyY3S7TAoI9OMGucSGlCKWbI774zCSnG6QmebG84n46zqktcUg4FggXA6m6PbnqTEkGysIXZR3ooQ2Sry/365NBzFG5MVb9TL1+iwhkopgA/KZzPtglZ7TxExxWCFQ3uK8pIo9kepZa21+uNjfRQDdwESZvMnAf63T907G2l0peFjcQYbrUF5wFekTiK07T9C+MRPiE97Wi0bRwOEqoahQ1DZHAa6Nkib5OvW0x7geWBkFBxiblAAoETRACOdYNlm5IxKgDT1rhMIsIAAERAgANJyX7ZDChD1MRn6kSJCTTGQWBaKBfyBCCrAUPUjxaAKOoQFYnCgfUZjJhAnAANzZGIWLLKBhAYiEN5MaJdQCnATlIFgArAGxhAaO5bFzHpQKtEccAAFRwDQrk+UC3kef5BkMES/mcCkhDAFYwAAF4hF5kViIlyWpZgmXZG5HkKgRzYwhqzqcKAUQgugWhwAA2gRAC6/lQNIzXZR57XOp1zURb1QA"}{Solution}

In addition, we will need the following lemma relating \Verb|eval| to \Verb|app|:

\begin{verbatim}
-- ∀ b, a ⬝ b ~ (eval a b)
theorem app_eval (a : Exp α) : ∀ b, (a ⬝ b) ~ (eval a b) :=
    by
    -- Hint: Induction on a
    sorry

\end{verbatim}



\hyperlink{"https://live.lean-lang.org/#codez=JYWwDg9gTgLgBAWQIYwBYBtgCMB0AVJAYxmEICgyBaSuAMSghAC45UYYwBnJgeh8M45gEQjgCmYnmIDunATwBMABiUBOHgBMAnlgBWYqDxBJgAOwAKSdCE5jKUMQDdgtjTjAaAZhTMaArsTAjmJwAKIAHmBwABSAjcBwLHhaYGIAlGQAPnBIYFEsEVHxgEmEYZFwxaWFmXDAGnAJleXVYuhiIA0VBU1mnsDhDQCMKnAARHCAuNSjcAC8AHyVODlgFFQ0ACK1pgDk8BpihOhIDnBoIQDKYjAQtTWmnDBIpoQhntAnqC5wWpdkvgEkwTghAgpkcUAa0S6sVScBKkLK0NhcHMDGWWSQnE4IjgAG8QmItnACYT8giAL4NYGg8HRaIhSYEmEMrZbGF0iYxAkc4mpdJZWoAfVanngePqpMKFPq4qBILBMXhYCEGiZRLZaWqgqgwAA5mxcSEJeUpdKWFT5ezJgVleq+XAHJ50NL6mKjbETTLzTS0py7ZwtO1nQaiSTGu7PXLverWUivZyY3S7TAoI9OMGucSGlCKWbI774zCSnG6QmebG84n46zqktcUg4FggXA6m6PbnqTEkGysIXZR3ooQ2Sry/365NBzFG5MVb9TL1+iwhkopgA/KZzPtglZ7TxExxWCFQ3uK8pIo9kepZa21+uNjfRQDdwESZvMnAf63T907G2l0peFjcQYbrUF5wFekTiK07T9C+MRPiE97Wi0bRwOEqoahQ1DZHAa6Nkib5OvW0x7geWBkFBxiblAAoETRACOdYNlm5IxKgDT1rhMIsIAAERAgANJyX7ZDChD1MRn6kSJCTTGQWBaKBfyBCCrCgY8WgCjqEBWJwoH1GYyYQJwADc2RiFiyygYQGIhDeTGiXUApwI5SBYAKwBsYQGhuWxcy6UCrRHHAABUcA0C53lAl57l+fpDBEn5nApIQwBWMAABeISeRFYgJUlKWYBl2Sue5CoEc2MIas6nCgFEILoFocAANoEQAun5UDSE1WXuW1zodU14U9ZhNC8VgAljkxa4SYRDbpBR9ZLDRQnRPWbpcXAo0CStHI9jhgnvrN0myfJemmP4SmmNklnWdkuRNnUhARVFPmzDFpgGQ2fl+HOEDoHUBF+VhAAS+kNAAwtADjEPVtz3I8JAoJlEWPHUz0JZD8nXbYNR1L5zpYXg2rOFY6NQFAmP1FZ2PIe0RGvfjNCE0E+Wk+TQA"}{Try it!}
\hyperlink{"https://live.lean-lang.org/#codez=JYWwDg9gTgLgBAWQIYwBYBtgCMB0AVJAYxmEICgyBaSuAMSghAC45UYYwBnJgeh8M45gEQjgCmYnmIDunATwBMABiUBOHgBMAnlgBWYqDxBJgAOwAKSdCE5jKUMQDdgtjTjAaAZhTMaArsTAjmJwAKIAHmBwABSAjcBwLHhaYGIAlGQAPnBIYFEsEVHxgEmEYZFwxaWFmXDAGnAJleXVYuhiIA0VBU1mnsDhDQCMKnAARHCAuNSjcAC8AHyVODlgFFQ0ACK1pgDk8BpihOhIDnBoIQDKYjAQtTWmnDBIpoQhntAnqC5wWpdkvgEkwTghAgpkcUAa0S6sVScBKkLK0NhcHMDGWWSQnE4IjgAG8QmItnACYT8giAL4NYGg8HRaIhSYEmEMrZbGF0iYxAkc4mpdJZWoAfVanngePqpMKFPq4qBILBMXhYCEGiZRLZaWqgqgwAA5mxcSEJeUpdKWFT5ezJgVleq+XAHJ50NL6mKjbETTLzTS0py7ZwtO1nQaiSTGu7PXLverWUivZyY3S7TAoI9OMGucSGlCKWbI774zCSnG6QmebG84n46zqktcUg4FggXA6m6PbnqTEkGysIXZR3ooQ2Sry/365NBzFG5MVb9TL1+iwhkopgA/KZzPtglZ7TxExxWCFQ3uK8pIo9kepZa21+uNjfRQDdwESZvMnAf63T907G2l0peFjcQYbrUF5wFekTiK07T9C+MRPiE97Wi0bRwOEqoahQ1DZHAa6Nkib5OvW0x7geWBkFBxiblAAoETRACOdYNlm5IxKgDT1rhMIsIAAERAgANJyX7ZDChD1MRn6kSJCTTGQWBaKBfyBCCrCgY8WgCjqEBWJwoH1GYyYQJwADc2RiFiyygYQGIhDeTGiXUApwI5SBYAKwBsYQGhuWxcy6UCrRHHAABUcA0C53lAl57l+fpDBEn5nApIQwBWMAABeISeRFYgJUlKWYBl2Sue5CoEc2MIas6nCgFEILoFocAANoEQAun5UDSE1WXuW1zodU14U9ZhNC8VgAljkxa4SYRDbpBR9ZLDRQnRPWbpcXAo0CStHI9jhgnvrN0myfJemmP4SmmNklnWdkuRNnUhARVFPmzDFpgGQ2fl+HOEDoHUBG5fs+XpSEz2fVVeWpYVj0ldNzazX5qBIICqDLiwtKidOqq7VNmMxHUkw9lxxFeosmIiO1Yi9KYIQDpGODJqmrDLgA/AKf7SgcYiBajiPIyEqMQnj0QE7NMK4xycN1ET0lUYsd109SOAOugQ7eRz9QOmYtOk4zdzM3AbMa/53PgrzzpiOERDwDDqAUPUVm2DUdS+c6sUQOD0qW9bcuCsKMD20CN3Ie0rvSu7nv1N7xByyrQA"}{Solution}



\section{Reification}

Now that we can evaluate out Monoid-expressions such that they're normalized at the Meta-level, how do we bring them back down to the object-level such that we don't change the "behavior" (wrt \Verb|~|)?
Well, for a given \Verb|e : Exp α|, we intuitively know that \Verb|eval e : Exp α → Exp α| will have the form:

\Verb|λ e' ↦ elem x₁ ⬝ (elem x₂ ⬝ ... ⬝ (elem xₙ ⬝ e'))|

So to reify it back down while retaining its \Verb|~|-behavior, simply apply \Verb|Exp.id| to the end:

\begin{verbatim}
def reify (f : Exp α → Exp α) : (Exp α) := f Exp.id

\end{verbatim}





\section{Nbe}

With our two main functions in place, we are finally ready to define our \Verb|nbe|-algorithm:

\begin{verbatim}
def nbe (e : Exp α) : Exp α := reify (eval e)

\end{verbatim}



What \Verb|nbe| does is first evaluate \Verb|e| so it gets normalized through function-composition, and then reify's it back into a canonical element of \Verb|[e]~|.
Thus, we can translate \Verb|a ~ b| into the decidable-problem \Verb|nbe a = nbe b| which we state as our main correctness-theorem:

\begin{verbatim}
theorem correctness (e e' : Exp α) : (e ~ e') ↔ (nbe e = nbe e') :=
    by sorry

\end{verbatim}



\hyperlink{"https://live.lean-lang.org/#codez=JYWwDg9gTgLgBAWQIYwBYBtgCMB0AVJAYxmEICgyBaSuAMSghAC45UYYwBnJgeh8M45gEQjgCmYnmIDunATwBMABiUBOHgBMAnlgBWYqDxBJgAOwAKSdCE5jKUMQDdgtjTjAaAZhTMaArsTAjmJwAKIAHmBwABSAjcBwLHhaYGIAlGQAPnBIYFEsEVHxgEmEYZFwxaWFmXDAGnAJleXVYuhiIA0VBU1mnsDhDQCMKnAARHCAuNSjcAC8AHyVODlgFFQ0ACK1pgDk8BpihOhIDnBoIQDKYjAQtTWmnDBIpoQhntAnqC5wWpdkvgEkwTghAgpkcUAa0S6sVScBKkLK0NhcHMDGWWSQnE4IjgAG8QmItnACYT8giAL4NYGg8HRaIhSYEmEMrZbGF0iYxAkc4mpdJZWoAfVanngePqpMKFPq4qBILBMXhYCEGiZRLZaWqgqgwAA5mxcSEJeUpdKWFT5ezJgVleq+XAHJ50NL6mKjbETTLzTS0py7ZwtO1nQaiSTGu7PXLverWUivZyY3S7TAoI9OMGucSGlCKWbI774zCSnG6QmebG84n46zqktcUg4FggXA6m6PbnqTEkGysIXZR3ooQ2Sry/365NBzFG5MVb9TL1+iwhkopgA/KZzPtglZ7TxExxWCFQ3uK8pIo9kepZa21+uNjfRQDdwESZvMnAf63T907G2l0peFjcQYbrUF5wFekTiK07T9C+MRPiE97Wi0bRwOEqoahQ1DZHAa6Nkib5OvW0x7geWBkFBxiblAAoETRACOdYNlm5IxKgDT1rhMIsIAAERAgANJyX7ZDChD1MRn6kSJCTTGQWBaKBfyBCCrCgY8WgCjqEBWJwoH1GYyYQJwADc2RiFiyygYQGIhDeTGiXUApwI5SBYAKwBsYQGhuWxcy6UCrRHHAABUcA0C53lAl57l+fpDBEn5nApIQwBWMAABeISeRFYgJUlKWYBl2Sue5CoEc2MIas6nCgFEILoFocAANoEQAun5UDSE1WXuW1zodU14U9ZhNC8VgAljkxa4SYRDbpBR9ZLDRQnRPWbpcXAo0CStHI9jhgnvrN0myfJemmP4SmmNklnWdkuRNnUhARVFPmzDFpgGQ2fl+HOEDoHUBG5fs+XpSEz2fVVeWpYVj0ldNzazX5qBIICqDLiwtKidOqq7VNmMxHUkw9lxxFeosmIiO1Yi9KYIQDpGODJqmrDLgA/AKf7SgcYiBajiPIyEqMQnj0QE7NMK4xycN1ET0lUYsd109SOAOugQ7eRz9QOmYtOk4zdzM3AbMa/53PgrzzpiOERDwDDqAUPUVm2DUdS+c6sUQOD0qW9bcuCsKMD20CN3Ie0rvSu7nv1N7xByyrFA7vaYjAJ4DXRLubpngi60noiTDEbu1ogWQCemFgtOGmG61QrLDjJ6nZUYVhZyoBAsjNkDGgucAmAwA1EC7iEa4EmQpzQChwJQA4xA05inIhsxhTZ4PIYwoAKYQxKX+IzHAm8r0dcmqbk9VwAAkp4nhCO9DCgQA7bcH1227Z3/MIl2P9KakaVp6A6UGpnmX5R2IRwDyyiLee62QBTUUbI9aiDknIvj8g7AK4IQphSgQKGBGCNBINCjQAAEvpFgABVJ2XolpWHorgrEk95K3zgN9V4f0d5l0TnXN6D8lB+SwoQ96JCnaC0eHURaAMqrQCgFoIAA"}{Try it!}
\hyperlink{"https://live.lean-lang.org/#codez=JYWwDg9gTgLgBAWQIYwBYBtgCMB0AVJAYxmEICgyBaSuAMSghAC45UYYwBnJgeh8M45gEQjgCmYnmIDunATwBMABiUBOHgBMAnlgBWYqDxBJgAOwAKSdCE5jKUMQDdgtjTjAaAZhTMaArsTAjmJwAKIAHmBwABSAjcBwLHhaYGIAlGQAPnBIYFEsEVHxgEmEYZFwxaWFmXDAGnAJleXVYuhiIA0VBU1mnsDhDQCMKnAARHCAuNSjcAC8AHyVODlgFFQ0ACK1pgDk8BpihOhIDnBoIQDKYjAQtTWmnDBIpoQhntAnqC5wWpdkvgEkwTghAgpkcUAa0S6sVScBKkLK0NhcHMDGWWSQnE4IjgAG8QmItnACYT8giAL4NYGg8HRaIhSYEmEMrZbGF0iYxAkc4mpdJZWoAfVanngePqpMKFPq4qBILBMXhYCEGiZRLZaWqgqgwAA5mxcSEJeUpdKWFT5ezJgVleq+XAHJ50NL6mKjbETTLzTS0py7ZwtO1nQaiSTGu7PXLverWUivZyY3S7TAoI9OMGucSGlCKWbI774zCSnG6QmebG84n46zqktcUg4FggXA6m6PbnqTEkGysIXZR3ooQ2Sry/365NBzFG5MVb9TL1+iwhkopgA/KZzPtglZ7TxExxWCFQ3uK8pIo9kepZa21+uNjfRQDdwESZvMnAf63T907G2l0peFjcQYbrUF5wFekTiK07T9C+MRPiE97Wi0bRwOEqoahQ1DZHAa6Nkib5OvW0x7geWBkFBxiblAAoETRACOdYNlm5IxKgDT1rhMIsIAAERAgANJyX7ZDChD1MRn6kSJCTTGQWBaKBfyBCCrCgY8WgCjqEBWJwoH1GYyYQJwADc2RiFiyygYQGIhDeTGiXUApwI5SBYAKwBsYQGhuWxcy6UCrRHHAABUcA0C53lAl57l+fpDBEn5nApIQwBWMAABeISeRFYgJUlKWYBl2Sue5CoEc2MIas6nCgFEILoFocAANoEQAun5UDSE1WXuW1zodU14U9ZhNC8VgAljkxa4SYRDbpBR9ZLDRQnRPWbpcXAo0CStHI9jhgnvrN0myfJemmP4SmmNklnWdkuRNnUhARVFPmzDFpgGQ2fl+HOEDoHUBG5fs+XpSEz2fVVeWpYVj0ldNzazX5qBIICqDLiwtKidOqq7VNmMxHUkw9lxxFeosmIiO1Yi9KYIQDpGODJqmrDLgA/AKf7SgcYiBajiPIyEqMQnj0QE7NMK4xycN1ET0lUYsd109SOAOugQ7eRz9QOmYtOk4zdzM3AbMa/53PgrzzpiOERDwDDqAUPUVm2DUdS+c6sUQOD0qW9bcuCsKMD20CN3Ie0rvSu7nv1N7xByyrFA7vaYjAJ4DXRLubpngi60noiTDEbu1ogWQCemFgtOGmG61QrLDjJ6nZUYVhZyoBAsjNkDGgucAmAwA1EC7iEa4EmQpzQChwJQA4xA05inIhsxhTZ4PIYwoAKYQxKX+IzHAm8r0dcmqbk9VwAAkp4nhCO9DCgQA7bcH1227Z3/MIl2P9KakaVp6A6UGpnmX5R2IRwDyyiLee62QBTUUbI9aiDknIvj8g7AK4IQphSgQKGBGCNBINYPzbIsDMEtk7ByCcU0pzlVlqTG8GCsFwNwdHG2kYlpWHogQ2hdRC44NvnAb6rw/o7zLonOub0H5KD5ijAYDRl6WgAsOPOMRSb+naKTLUuo2DGy1jTRR9M9ZplQFIo2gCUGsAGBIgWUj0b0jkeLUqQkfRcOJrdMALCnQhC4ZTamOtdEpn1gYw27NjGm1MZTaQ2oYAhEaqjXqnMTHm2lEjFGaM7EHkZDYvaJYOSOKofTZRnZciuPno4zx2sdFKz0QbIxzouY83ERbK2MdVFeW1HqGAQA"}{Solution}

With \Verb|correctness| in place, Lean can now instantly decide any \Verb|a ~ b| problem through reflexivity, i.e.:

\begin{verbatim}
-- (1 ⬝ 2) ⬝ ((0 ⬝ 0) ⬝ 3) ~ 1 ⬝ (2 ⬝ (3 ⬝ 0))
theorem example1'
        : (one.app two).app  ((zero.app zero).app three)
        ~ (one.app (two.app (three.app zero))) :=
  by
  /- Try this:
    exact (correctness ((one ⬝ two) ⬝ ((zero ⬝ zero) ⬝ three)) (one ⬝ (two ⬝ (three ⬝ zero)))).mpr rfl
  -/
  exact?

-- (0 ⬝ (1 ⬝ 0)) ⬝ ((0 ⬝  2) ⬝ (3 ⬝ 0)) ~ 1 ⬝ (2 ⬝ (3 ⬝ 0))
theorem example2'
  : (zero.app (one.app zero)).app ((zero.app two).app (three.app zero))
  ~ (one.app (two.app (three.app zero))) :=
  by
  /- Try this:
    exact (correctness ((zero ⬝ (one ⬝ zero)) ⬝ ((zero ⬝ two) ⬝ (three ⬝ zero)))
      (one ⬝ (two ⬝ (three ⬝ zero)))).mpr rfl
  -/
  exact?

-- (1 ⬝ 2) ⬝ ((0 ⬝ 0) ⬝ 3) ~ (0 ⬝ (1 ⬝ 0)) ⬝ ((0 ⬝  2) ⬝ (3 ⬝ 0))
theorem example3'
  : (one.app two).app  ((zero.app zero).app three)
  ~ (zero.app (one.app zero)).app ((zero.app two).app (three.app zero)) :=
  by
  /- Try this:
    exact (correctness ((one ⬝ two) ⬝ ((zero ⬝ zero) ⬝ three))
      ((zero ⬝ (one ⬝ zero)) ⬝ ((zero ⬝ two) ⬝ (three ⬝ zero)))).mpr rfl
  -/
  exact?

\end{verbatim}






\chapter{GodelT-Rewriting}



\section{GodelT Expressions and rewriting}

We now move on to a Combinatory-version of Godel's System T and implement NbE for it as well.
Here we will be using an Intrinsic encoding (aka "typing ala Church") meaning all Expressions will be well-typed and we won't need an additional "Derivation" type.
First, we define our Simple-Types:

\begin{verbatim}
inductive Ty : Type
| Nat : Ty
| arrow : Ty → Ty → Ty
open Ty
infixr : 100 " ⇒' " => arrow

\end{verbatim}



Here our base-type is the Natural Numbers while \Verb|arrow| lets us form functions between Simple Types.
Our Expressions are:

\begin{verbatim}
inductive ExpT : Ty → Type
| K {a b : Ty}     :  ExpT (a ⇒' b ⇒' a)
| S {a b c : Ty}   :  ExpT ((a ⇒' b ⇒' c) ⇒' (a ⇒' b) ⇒' (a ⇒' c))
| App {a b : Ty}   :  ExpT (a ⇒' b) → ExpT a → ExpT b
| zero             :  ExpT .Nat
| succ             :  ExpT (.Nat ⇒' .Nat)
| recN {a : Ty}    :  ExpT (a ⇒' (.Nat ⇒' a ⇒' a) ⇒' .Nat ⇒' a)
open ExpT
infixl : 100 " ⬝ " => App

\end{verbatim}



That is, our Expressions are either:

\begin{itemize}
\item Combinators \Verb|K| and \Verb|S|\item Applying two expressions together: \Verb|e₁ ⬝ e₂|\item The Natural number \Verb|zero|\item The successor function \Verb|succ|\item The recursor for Natural numbers: \Verb|recN|
\end{itemize}


From this, we get the following rewriting relation for \Verb|~|:

\begin{verbatim}
inductive convrT : (ExpT α) → (ExpT α) → Prop
| refl  : convrT (e) (e)
| sym   : convrT (e) (e') → convrT (e') (e)
| trans : convrT (e) (e') → convrT (e') (e'') → convrT (e) (e'')
| K     : convrT (K ⬝ x ⬝ y) (x)
| S     : convrT (S ⬝ x ⬝ y ⬝ z) (x ⬝ z ⬝ (y ⬝ z))
| app   : convrT (a) (b) → convrT (c) (d) → convrT (a ⬝ c) (b ⬝ d)
| recN_zero : convrT (recN ⬝ e ⬝ f ⬝ .zero) (e)
--| recN_succ : convrT (recN ⬝ e ⬝ f ⬝ (.succ ⬝ n)) (f ⬝ n ⬝ (recN ⬝ e ⬝ f ⬝ n))
infix : 100 " ~ " => convrT

\end{verbatim}








\chapter{Index}






\end{document}
